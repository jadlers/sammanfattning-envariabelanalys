\documentclass{article}
\usepackage[utf8]{inputenc}
\setlength{\parindent}{0pt}

\title{Envariabelanalys\\Sammanfattning av definitioner och satser}
\author{Jacob Adlers}

\usepackage{natbib}
\usepackage{graphicx}

\begin{document}

\maketitle
\newpage

% #### FUNKTIONER ####
\section{Funktioner}
\subsection{Definition}
En funktion $f$ är en regel som för varje element i en mängd, definitionsmängden av $f$, tilldelar ett unikt element i värdemängden av $f$.

\subsection{Sats (Bevis sid 51)}
$cos(s-t)=cos(s)cos(t)+sin(s)sin(t)$

\subsection{Sats}
Om $f(x)$ är både jämn och udda då är $f(x)=0 \quad \forall x $

\subsection{Sats}
Om $p(x)$ är ett polynom och $p(a)=0$ så finns det ett polynom $q(x)$ sådant att $p(x)=q(x)(x-a)$


\end{document}
