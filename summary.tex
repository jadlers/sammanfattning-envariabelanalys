\documentclass{article}
\usepackage[utf8]{inputenc}
\setlength{\parindent}{0pt}

\title{Envariabelanalys\\Sammanfattning av definitioner och satser}
\author{Jacob Adlers}

\usepackage{natbib}
\usepackage{graphicx}

\newcommand\varpm{\mathbin{\vcenter{\hbox{%
  \oalign{\hfil$\scriptstyle+$\hfil\cr
          \noalign{\kern-.3ex}
          $\scriptscriptstyle({-})$\cr}%
}}}}
\newcommand\varmp{\mathbin{\vcenter{\hbox{%
  \oalign{$\scriptstyle({+})$\cr
          \noalign{\kern-.3ex}
          \hfil$\scriptscriptstyle-$\hfil\cr}%
}}}}

\begin{document}

\maketitle
\newpage

% #### FUNKTIONER ####
\section{Funktioner}
\subsection{Definition}
En funktion $f$ är en regel som för varje element i en mängd, definitionsmängden av $f$, tilldelar ett unikt element i värdemängden av $f$.

\subsection{Sats (Bevis sid 51)}
$cos(s-t)=cos(s)cos(t)+sin(s)sin(t)$

\subsection{Sats}
Om $f(x)$ är både jämn och udda då är $f(x)=0 \quad \forall x $

\subsection{Sats}
Om $p(x)$ är ett polynom och $p(a)=0$ så finns det ett polynom $q(x)$ sådant att $p(x)=q(x)(x-a)$

% #### GRÄNSVÄRDEN ####
\section{Gränsvärden}
\subsection{Definition}
Vi säger att $f(x)$ går mot $L\in \mathcal{R}$ när $x$ går mot oändligheten ($f(x) \to L$ då $x \rightarrow \infty$ $\lim\limits_{x \to\infty} f(x)=L$).\\
Det gäller om det $\forall\varepsilon>0$ existerar ett $R_{\varepsilon}$ sådant att om $x>R_{\varepsilon}$ så $|f(x)-L|<\varepsilon$

\subsection{Definition}
Vi säger att en funktion $f(x)$ går mot $L$ då $x$ går mot $a$ om det $\forall\varepsilon>0$ existerar ett $\delta_{\varepsilon}>0$ sådant att $0<|x-a|<\delta_{\varepsilon}$. Det medför att $|f(x)-L|<\varepsilon$

\subsection{Definition}
$\lim\limits_{x\to a^{+}} f(x) = L$

% #### KONTINUITET ####
\section{Kontinuitet}
\subsection{Definition}
Vi säger att en funktion $f(x)$ är kontinuerlig i en inre punkt $c$ av sitt definitionsområde om $\lim\limits_{x\to c}f(x)=f(c)$

\subsection{Definition}
Vi säger att $f(x)$ är vänster/(höger)-kontinuerlig i en punkt $c$ om: $\lim\limits_{x\to c^{-}} f(x)=f(c) \quad (\lim\limits_{x\to c^{+}} f(x)=f(c))$

\subsection{Sats}
Om $f(x)$ och $g(x)$ är kontinuerliga så kommer $f(x)+g(x)$, $f(x)-g(x)$, $f(x)g(x)$ och $f(g(x))$ att vara kontinuerliga där de är definerade.

\subsection{Sats}
Om $f(x)$ är kontinuerlig på ett slutet och begränsat intervall  $[a,b]$ då kommer det att finnas två punkter $p,q\in [a,b]$ sådant att $f(p)\leq f(x)\leq f(q) \quad \forall x\in [a,b]$

\subsection{Sats om mellanliggande värden}
Om $f(x)$ är kontinuerlig på $[a,b]$ och om $s$ ligger mellan $f(a)$ och $f(b)$ då finns det ett $x\in [a,b]$ sådant att $f(x)=s$

% #### DERIVATA ####
\section{Derivata}
\subsection{Definition}
Vi säger att derivatan av en funktion $f(x)$ ges av $\lim\limits_{h\to  0}\frac{f(x+h)-f(x)}{h}$ om gränsvärdet existerar.

\subsection{Definition}
Om $f(x)$ är deriverbar i punkten $x_{0}$ så är linjen $y=f'(x_{0})(x-x_{0})+f(x_{0})$ tangenten till $f(x)$ i $x_{0}$.

\subsection{Sats}
Om $f(x)$ och $g(x)$ är deriverbara så gäller följande:

\begin{enumerate}
   \item $D(f(x)\varpm g(x))$ (Summaregeln)
   \item $D(f(x)g(x))=f'(x)g(x)+f(x)g'(x)$ (Produktregeln)
   \item $D(\frac{f(x)}{g(x)})=\frac{f'(x)g(x)-f(x)g'(x)}{(g(x))^2}$ (Kvotregeln) Om $g(x)\neq 0$
\end{enumerate}

\subsection{Sats}
Om en funktion $g(x)$ är deriverbar i $x_{0}$ så är $g(x)$ kontinuerlig i $x_{0}$. Alltså,\\
$g(x)$ deriverbar $\Rightarrow g(x)$ kontinuerlig.

\subsection{Sats}
\begin{enumerate}
   \item $Dx=1$
   \item $Dx^r=rx^{r-1} \quad r \in \mathcal{R}$
   \item $D\sin(x)=\cos(x)$
   \item $D\cos(x)=-\sin(x)$
\end{enumerate}

\subsection{Definition}
Om $f(x)$ är en funktion definerad på ett intervall $I$ så säger vi att $f(x)$ är:

\begin{enumerate}
   \item Strängt växande på $I$ om $\forall x_{1},x_{2}\in I \quad x_{2}>x_{1} \Rightarrow f(x_{2})>f(x_{1})$
   \item Växande på $I$ om $\forall x_{1},x_{2}\in I \quad x_{2}>x_{1} \Rightarrow f(x_{2})\geq f(x_{1})$
   \item Strängt avtagande på $I$ om $\forall x_{1},x_{2}\in I \quad x_{2}>x_{1} \Rightarrow f(x_{2})<f(x_{1})$
   \item Avtagande på $I$ om $\forall x_{1},x_{2}\in I \quad x_{2}>x_{1} \Rightarrow f(x_{2})\leq f(x_{1})$
\end{enumerate}

\subsection{Medelvärdessatsen}
Om $f(x)$ är kontinuerlig på ett intervall $[a,b]$ och $f(x)$ är deriverbar på $(a,b)$ då finns en punkt $c\in (a,b)$ så att: $\frac{f(b)-f(a)}{b-a}=f'(c)$

\subsection{Följdsats till medelvärdessatsen}
Antag att $f'(x)>0$ på $(a,b)$. Då är $f(x)$ strängt växande på samma intervall.

\end{document}
