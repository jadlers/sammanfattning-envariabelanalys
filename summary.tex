\documentclass{article}
\usepackage[utf8]{inputenc}
\setlength{\parindent}{0pt}

\title{Envariabelanalys\\Sammanfattning av definitioner och satser}
\author{Jacob Adlers}

\usepackage{natbib}
\usepackage{graphicx}

\newcommand\varpm{\mathbin{\vcenter{\hbox{%
  \oalign{\hfil$\scriptstyle+$\hfil\cr
          \noalign{\kern-.3ex}
          $\scriptscriptstyle({-})$\cr}%
}}}}
\newcommand\varmp{\mathbin{\vcenter{\hbox{%
  \oalign{$\scriptstyle({+})$\cr
          \noalign{\kern-.3ex}
          \hfil$\scriptscriptstyle-$\hfil\cr}%
}}}}

\begin{document}

\maketitle
\newpage

% #### FUNKTIONER ####
\section{Funktioner}
\subsection{Definition}
En funktion $f$ är en regel som för varje element i en mängd, definitionsmängden av $f$, tilldelar ett unikt element i värdemängden av $f$.

\subsection{Definition}
Om $f(x)$ är en funktion definerad på ett intervall $I$ så säger vi att $f(x)$ är:

\begin{enumerate}
   \item Strängt växande på $I$ om $\forall x_{1},x_{2}\in I \quad x_{2}>x_{1} \Rightarrow f(x_{2})>f(x_{1})$
   \item Växande på $I$ om $\forall x_{1},x_{2}\in I \quad x_{2}>x_{1} \Rightarrow f(x_{2})\geq f(x_{1})$
   \item Strängt avtagande på $I$ om $\forall x_{1},x_{2}\in I \quad x_{2}>x_{1} \Rightarrow f(x_{2})<f(x_{1})$
   \item Avtagande på $I$ om $\forall x_{1},x_{2}\in I \quad x_{2}>x_{1} \Rightarrow f(x_{2})\leq f(x_{1})$
\end{enumerate}

\subsection{Definition}
Vi säger att $f(x)$ är injektiv om $f(x_{1})=f(x_{2}\Rightarrow x_{1}=x_{2})$.

\subsection{Sats (Bevis sid 51)}
$cos(s-t)=cos(s)cos(t)+sin(s)sin(t)$

\subsection{Sats}
Om $f(x)$ är både jämn och udda då är $f(x)=0 \quad \forall x $

\subsection{Sats}
Om $p(x)$ är ett polynom och $p(a)=0$ så finns det ett polynom $q(x)$ sådant att $p(x)=q(x)(x-a)$

% #### GRÄNSVÄRDEN ####
\section{Gränsvärden}
\subsection{Definition}
Vi säger att $f(x)$ går mot $L\in \mathcal{R}$ när $x$ går mot oändligheten ($f(x) \to L$ då $x \rightarrow \infty$ $\lim\limits_{x \to\infty} f(x)=L$).\\
Det gäller om det $\forall\varepsilon>0$ existerar ett $R_{\varepsilon}$ sådant att om $x>R_{\varepsilon}$ så $|f(x)-L|<\varepsilon$

\subsection{Definition}
Vi säger att en funktion $f(x)$ går mot $L$ då $x$ går mot $a$ om det $\forall\varepsilon>0$ existerar ett $\delta_{\varepsilon}>0$ sådant att $0<|x-a|<\delta_{\varepsilon}$. Det medför att $|f(x)-L|<\varepsilon$

% \subsection{Definition??}
% $\lim\limits_{x\to a^{+}} f(x) = L$

\subsection{Sats}
\begin{enumerate}
   \item $\lim\limits_{x\to\infty}\frac{\log_a(x)}{x^\alpha}=0 \quad \forall\alpha > 0$
   \item $\lim\limits_{x\to\infty}\frac{x^\alpha}{a^x}\quad \forall a>1,\alpha \in \mathcal{R}$
\end{enumerate}

% #### KONTINUITET ####
\section{Kontinuitet}
\subsection{Definition}
Vi säger att en funktion $f(x)$ är kontinuerlig i en inre punkt $c$ av sitt definitionsområde om $\lim\limits_{x\to c}f(x)=f(c)$

\subsection{Definition}
Vi säger att $f(x)$ är vänster/(höger)-kontinuerlig i en punkt $c$ om: $\lim\limits_{x\to c^{-}} f(x)=f(c) \quad (\lim\limits_{x\to c^{+}} f(x)=f(c))$

\subsection{Sats}
Om $f(x)$ och $g(x)$ är kontinuerliga så kommer $f(x)+g(x)$, $f(x)-g(x)$, $f(x)g(x)$ och $f(g(x))$ att vara kontinuerliga där de är definerade.

\subsection{Sats}
Om $f(x)$ är kontinuerlig på ett slutet och begränsat intervall  $[a,b]$ då kommer det att finnas två punkter $p,q\in [a,b]$ sådant att $f(p)\leq f(x)\leq f(q) \quad \forall x\in [a,b]$

\subsection{Sats om mellanliggande värden}
Om $f(x)$ är kontinuerlig på $[a,b]$ och om $s$ ligger mellan $f(a)$ och $f(b)$ då finns det ett $x\in [a,b]$ sådant att $f(x)=s$

% #### DERIVATA ####
\section{Derivata}
\subsection{Definition}
Vi säger att derivatan av en funktion $f(x)$ ges av $\lim\limits_{h\to  0}\frac{f(x+h)-f(x)}{h}$ om gränsvärdet existerar.

\subsection{Definition}
Om $f(x)$ är deriverbar i punkten $x_{0}$ så är linjen $y=f'(x_{0})(x-x_{0})+f(x_{0})$ tangenten till $f(x)$ i $x_{0}$.

\subsection{Definition}
Vi säger att $c$ är ett lokalt max/(min) om det finns $\alpha,\beta$ sådana att:\\ $f(c)\geq f(x)\quad (f(c)\leq f(x)),\quad \forall x \in (\alpha, \beta)$.\\
(LÄGG IN KOPPLING TILL DERIVATA (föreläsning 12))

\subsection{Sats}
Om $f(x)$ och $g(x)$ är deriverbara så gäller följande:

\begin{enumerate}
   \item $D(f(x)\varpm g(x))$ (Summaregeln)
   \item $D(f(x)g(x))=f'(x)g(x)+f(x)g'(x)$ (Produktregeln)
   \item $D(\frac{f(x)}{g(x)})=\frac{f'(x)g(x)-f(x)g'(x)}{(g(x))^2}$ (Kvotregeln) Om $g(x)\neq 0$
\end{enumerate}

\subsection{Sats}
Om en funktion $g(x)$ är deriverbar i $x_{0}$ så är $g(x)$ kontinuerlig i $x_{0}$. Alltså,\\
$g(x)$ deriverbar $\Rightarrow g(x)$ kontinuerlig.

\subsection{Sats}
\begin{enumerate}
   \item $Dx=1$
   \item $Dx^r=rx^{r-1} \quad r \in \mathcal{R}$
   \item $D\sin(x)=\cos(x)$
   \item $D\cos(x)=-\sin(x)$
\end{enumerate}

\subsection{Medelvärdessatsen}
Om $f(x)$ är kontinuerlig på ett intervall $[a,b]$ och $f(x)$ är deriverbar på $(a,b)$ då finns en punkt $c\in (a,b)$ så att: $\frac{f(b)-f(a)}{b-a}=f'(c)$

\subsection{Följdsats till medelvärdessatsen}
Antag att $f'(x)>0$ på $(a,b)$. Då är $f(x)$ strängt växande på samma intervall.

\newpage
\section{Differentialekvationer}
\subsection{Definition}
Vi säger att $y_{h}(x)$ är en homogen lösning om $y''_{h}(x)+ay'_{h}(x)+by_{h}(x)=0$

\subsection{Definition}
Vi säger att $y_{p}(x)$ är en partikulärlösning till $y''(x)+ay'(x)+by(x)=f(x)$ om $y_{p}(x)$ är någon funktion som uppfyller ekvationen.

\subsection{Lösningsstrategi}
$y''(x)+ay'(x)+by(x)=f(x),\quad y(x_{0})=\alpha,\quad y'(x_{0})=\beta$
\begin{enumerate}
   \item Hitta alla homogena lösningar $y_{h}(x)$
   \begin{enumerate}
      \item Hitta rötterna till det karakteristiska polynomet\\
      $r^2+ar+b=0\Rightarrow r_{1},r_{2}=-\frac{a}{2}\pm \sqrt{(\frac{a}{2})^2-b}$
      \item
      \begin{enumerate}
         \item Om $r_{1}\neq r_{2}$ och $r_{1},r_{2}\in\mathcal{R}$ då är:\\
         $y_{h}(x)=Ce^{r_{1}x}+De^{r_{2}x}$ för några $C,D\in\mathcal{R}$
         \item Om $r_{1}=r_{2}$ och $r_{1},r_{2}\in\mathcal{R}$ då är:\\
         $y_{h}(x)=Cxe^{r_{1}x}+De^{r_{1}x}$ för några $C,D\in\mathcal{R}$
         \item Om $r_{1},r_{2}=k\pm i\omega$ då är:\\
         $y_{h}(x)=Ce^{kx}\sin(\omega x)+De^{kx}\cos(\omega x)$ för några $C,D\in\mathcal{R}$
      \end{enumerate}
   \end{enumerate}
   \item Om $f(x)\neq 0$ gissa en partikulärlösning enligt tabellen och bekräfta den.
   \begin{table}[!h]
   \centering
   \begin{tabular}{|r|l|}
   \hline
   $f(x)$                            & Gissning av $y_{p}$                                         \\ \hline
   Konstant                          & $y_{p}=$ Konstant                                            \\ \hline
   Polynom                           & $y_{p}=$ Polynom av samma grad                               \\ \hline
   $e^{\lambda x}$                   & $y_{p}=Ae^{\lambda x}$                                      \\ \hline
   $e^{\mu x}\sin(\lambda x)$ eller $e^{\mu x}\cos(\lambda x)$        & $y_{p}=Ae^{\mu x}\sin(\lambda x)+Be^{\mu x}\cos(\lambda x)$ \\ \hline
   \end{tabular}
   \end{table}

   Om gissningen är en homogen lösning så multiplicera den partikulära lösningen med $x$ alternativt $x^2$ om multiplikation med $x$ också är en homogen lösning. Kombination av $f(x)$ ger kombination av gissningar enligt tabellen ovan.
   \item Ansätt $y(x)=y_{h}(x)(+y_{p}(x))$ och beräkna $C$ och $D$ genom att använda initialdatan. Vi får då ett linjärt ekvationsystem på formen:\\
   $\left\{
   \begin{array}{l}
       y(x_{0})=\alpha \\
       y'(x_{0})=\beta
   \end{array}
   \right.$
\end{enumerate}

\newpage
\section{Taylorpolynom}
Det finns fler satser som leder fram till Taylorpolynomet under föreläsning 12.
\subsection{Defintion}
Om $f(x)$ är $n$ gånger deriverbar i punkten $a$, då är Taylorpolynomet av ordning $n$ till funktionen $f$ i punkten $a$:\\
$P_{n}(x)=f(a)+f'(a)(x-a)+\frac{f''(a)}{2!}(x-a)^2+\cdots+\frac{f^{(n)}(a)}{n!}(x-a)^n$

\subsection{Taylors sats}
Om $f(x)$ är $(n+1)$ gånger deriverbar i något öppet intervall kring $a$ och $P_{n}$ är Taylorpolynomet i $a$. Då gäller att $f(x)-P_{n}(x)=\frac{f^{n+1}(s)}{(n+1)!}(x-a)^{(n+1)}$ för något $s$ mellan $a$ och $x$.

\section{Integraler}
\subsection{Definition}
Om det finns exakt ett tal $I$ sådant att $\exists f(x)$ definerad på $[a,b]$, för varje indelning $p$ säger vi att:\\
${\displaystyle L(f,p)\leq I \leq U(f,p) \Rightarrow \int_{a}^{b}f(x)dx}$

\subsection{Sats}
Om $f(x)$ är kontinuerlig på $[a,b]$ så är $f(x)$ integrerbar, dvs ${\displaystyle\int_{a}^{b}f(x)dx}$ existerar.

\subsection{Analysens huvudsats}
Antag att $f(x)$ är kontinuerlig på ett intervall $I$ (öppet eller slutet) och $a\in I$. Då:
\begin{enumerate}
   \item Om ${\displaystyle F(x)=\int_{a}^{x}f(t)dt}$ så kommer $F'(x)=f(x)\quad \forall x\in I$
   \item Om $G(x)$ är en primitiv funktion till $f(x)$, dvs ($G'(x)=f(x)$), och $b\in I$ så kommer ${\displaystyle\int_{a}^{b}f(x)dx=G(b)-G(a)}$
\end{enumerate}

\subsection{Definition}
Ytan mellan $f(x)$ och $g(x)$ då $a\leq x\leq b$ ges av ${\displaystyle \int_{a}^{b} |f(x)-g(x)|dx}$

\section{Integrationsmetoder}
\subsection{Variabelsubstitution}
\subsubsection{Förklaring}
Om $F(x)$ är en primitiv till $f(x)$ och $g(x)$ är deriverbar. Då kommer $F(g(x))$ vara en primitiv funktion till $f(g(x))g'(x)$. Variabelsubstitution används för att beräkna integraler på form ${\displaystyle\int_{a}^{b}f(g(x))g'(x)dx=F(g(x))\bigg\vert_{a}^{b}}$

\subsubsection{Exempel}
${\displaystyle \int_{1}^{2}\frac{x}{(x^2+4)^2}dx = \frac{1}{2}\int_{1}^{2}\frac{2x}{(x^2+4)^2}dx = \left\{
\begin{array}{l}
    f(x) = \frac{1}{x^2}   \\
    g(x) = x^2+4           \\
    g'(x)= 2x
\end{array}
\right\} = \left\{
\begin{array}{l}
    t=x^2+4,\quad dt=2xdx \\
    x=1 \Rightarrow t=5 \\
    x=2 \Rightarrow t=8
\end{array}
\right\}}$ \\
${\displaystyle= \frac{1}{2}\int_{5}^{8}\frac{1}{t^2}dt = \frac{1}{2}(-\frac{1}{t})\bigg\vert_{5}^{8}=-\frac{1}{2}\frac{1}{8}+\frac{1}{2}\frac{1}{5}=\frac{1}{10}-\frac{1}{16}=\frac{8}{80}-\frac{5}{80}=\frac{3}{80}}$

\subsection{Partiell integration}
\subsubsection{Förklaring}
Partiell integration följer av derivatans produktregel. Enligt produktregeln så ska $D(f(x)g(x))=f'(x)g(x)+f(x)g'(x)$. Hittar vi sedan den primitiva funktionen till ${\displaystyle f(x)g(x)=\int f'(x)g(x)dx + \int f(x)g'(x)dx}$ som kan skrivas om till:\\
${\displaystyle\int_{a}^{b}f(x)g(x)dx=f(x)G(x)\bigg\vert_{a}^{b} - \int_{a}^{b}f'(x)G(x)dx}$

\subsubsection{Exempel}
${\displaystyle \int_{0}^{\frac{\pi}{2}}x\sin(x)dx=\left\{\begin{array}{l}
   f(x)=x   \\
   g(x)=sin(x)
\end{array}
\right\} = x \cdot-\cos(x)\bigg\vert_{0}^{\frac{\pi}{2}} + \int_{0}^{\frac{\pi}{2}}\cos(x)dx}$ \\
$= -x\cos(x)\bigg\vert_{0}^{\frac{\pi}{2}} + \sin(x)\bigg\vert_{0}^{\frac{\pi}{2}} = \sin(x)\bigg\vert_{0}^{\frac{\pi}{2}}=1$

\subsection{Partialbråksuppdelning}
\subsubsection{Förklaring}
\subsubsection{Exempel}

\end{document}
